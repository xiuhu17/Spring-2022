\documentclass{article}
\usepackage{indentfirst}
\usepackage{geometry}
\usepackage{ntheorem}
\usepackage{amsmath}
\usepackage{amssymb}
\newtheorem*{proposition}{Proposition}
\newtheorem*{definition}{Definition}
\newtheorem*{corrolary}{Corrolary}
\newtheorem*{consider}{Consider}
\newtheorem*{theorem}{Theorem}
\newtheorem*{suppose}{Suppose}
\newtheorem*{notice}{Notice}
\newtheorem*{define}{Define}
\newtheorem*{denote}{Denote}
\newtheorem*{lemma}{Lemma}
\newtheorem*{claim}{Claim}
\newtheorem*{proof}{Proof}
\newtheorem*{case}{Case}
%\begin{equation*} \end{equation*}%
%\begin{equation} \end{equation}%
%\begin{split} \end{split}%
%\begin{cases} \end{cases}%
%\subsubsection*{(a)}%
%\subsection*{(a)}%
%\textbf{}%
%\textit{}%
%\noindent%
%\paragraph{}%
%\page{}%
%\\%
%$\langle \rangle$%
%$\| \|$%
% \begin{equation*}
%     \begin{split}
%         & (\lfloor \frac{t}{d_3} \rfloor - 2p) + f((t \mod d_3) + 2 d_3 p, 2) \\
%         \rightarrow & (\lfloor \frac{t}{d_3} \rfloor - 2p) 
%         + \lfloor \frac{(t \mod d_3) + 2 d_3 p}{d_2} \rfloor 
%         + f(((t \mod d_3) + 2 d_3 p) \mod d_2, 1) \\
%         = & (\lfloor \frac{t}{25} \rfloor - 2p) 
%         + \lfloor \frac{(t \mod 25) + 2 \cdot 25 \cdot p}{10} \rfloor 
%         + f(((t \mod 25) + 2 \cdot 25 \cdot p) \mod 10, 1) \\
%         = & (\lfloor \frac{t}{25} \rfloor - 2p) 
%         + \lfloor \frac{t \mod 25}{10} \rfloor + 5 p 
%         + f((t \mod 25) \mod 10, 1) \\
%         = & \lfloor \frac{t}{25} \rfloor
%         + \lfloor \frac{t \mod 25}{10} \rfloor + 3 p 
%         + f((t \mod 25) \mod 10, 1) 
%     \end{split}
%     \end{equation*}

% \begin{equation}
%     \begin{split}
%     (a + b)^4
%       &= (a + b)^2 (a + b)^2     \\
%       &= (a^2 + 2ab + b^2)
%         (a^2 + 2ab + b^2)       \\
%       &= a^4 + 4a^3b + 6a^2b^2 + 4ab^3 + b^4
%     \end{split}
%    \end{equation}

\geometry{a4paper}
\title{Homework11}
\author{Zhihao Wang} 
\date{04/22/2022}
\begin{document}
\maketitle 

\subsection*{4.5.4}
\subsubsection*{(a)}
\begin{denote}
    The $ \left(\begin{bmatrix}
        1 \\ 0
    \end{bmatrix}, \begin{bmatrix}
        1 \\ 1
    \end{bmatrix} \right) $ as $B_v$
\end{denote}
The $ith$ column of $[R]_{B_v}$ is $[Rv_i]_{B_v}$.
\begin{equation*}
    \begin{split}
        &[Rv_1]_{B_v} = \begin{bmatrix}
            1 \\ 0
        \end{bmatrix}_{B_v} = \begin{bmatrix}
            1 \\ 0
        \end{bmatrix} \\
        &[Rv_2]_{B_v} = \begin{bmatrix}
            1 \\ -1
        \end{bmatrix}_{B_v} = \begin{bmatrix}
            2 \\ -1
        \end{bmatrix} \\
        \Rightarrow & \begin{bmatrix}
            1 & 2 \\ 0 & -1
        \end{bmatrix}
    \end{split} 
\end{equation*}

\subsubsection*{(b)}
\begin{proof}
    By \textbf{Proposition 4.31} and \textbf{Definition of Orthogonal Matrix}, we know that if the columns are not \textbf{Othonomal}, then the matrix must
    not be orthogonal.
    \begin{equation*}
        \begin{split}
            \langle A_1, A_2 \rangle = 2 \neq 0
        \end{split}
    \end{equation*}
    Therefore, $A$ is not orthogonal matrix.
\end{proof}


\subsubsection*{(c)}
\begin{proof}
    Because \textbf{Proposition 4.30} says, \textbf{Suppose} $B_v, B_w$ \textbf{are orthonormal basis of} $\mathbb{V, W}$. But $ \left(\begin{bmatrix}
        1 \\ 0
    \end{bmatrix}, \begin{bmatrix}
        1 \\ 1
    \end{bmatrix} \right) $ itself is not an orthonomal basis.
\end{proof}

\subsection*{4.5.6}
\begin{proof}
    Since $\mathbb{C}_{2 \pi}(\mathbb{R})$ is equipped with innerporduct, so it is both an innerproduct space and a normed space. \\
    Suppose $\forall f(x) \in \mathbb{C}_{2 \pi}(\mathbb{R})$, which means $f(x)$ is a continuous $2 \pi$ periodic function $\Rightarrow$
    $\exists g(x) = f(x - t) \in \mathbb{C}_{2 \pi}(\mathbb{R}), T(g)(x) = f(x + t)$. $T$ is a surjective lineaer map.
    \begin{denote}
        \begin{equation*}
            \begin{split}
                & \int_{a}^{b} f(x) \overline{f(x)} \,d(x) = F(b) - F(a) = \int_{a + 2 \pi}^{b + 2 \pi} f(x) \overline{f(x)} \,d(x) = F(b + 2 \pi ) - F(a+ 2 \pi) \\
                \Rightarrow & F(b + 2 \pi) - F(b) =  F(a + 2 \pi) - F(a)
            \end{split}
        \end{equation*}
    \end{denote}

    \textbf{So we get}
    \begin{equation*}
        \begin{split}
            \|(Tf)(x)\|   \int_{0}^{2 \pi} (Tf)(x) \overline{(Tf)(x)} \,dx & = \int_{0}^{2 \pi} f(x+t) \overline{f(x+t)} \,dx \\
                        & = \int_{0}^{2 \pi} f(x+t) \overline{f(x+t)} \,d(x + t) \\
                        & = \int_{t}^{t + 2 \pi} f(x') \overline{f(x')} \,d(x') \\
                        & = F(t + 2 \pi) - F(t) \\
                        & = F(0 + 2 \pi) - F(0) \\
                        & = \|(f)(x)\|
        \end{split}
    \end{equation*}
    So, it is an isometry.
\end{proof}

\subsection*{4.5.8}
\subsection*{4.5.10}
\subsection*{4.5.14}
\subsubsection*{(a)}
\begin{proof}
    By \textbf{Proposition 4.30}, we know that if $U$ is unitary, then all columns of $U$ is orthonomal. By \textbf{Theorem 4.3}, we know that all columns are independent. 
    By \textbf{Theorem 3.28}, we know that the columns are orthonomal basis. By \textbf{Corrolary 4.30}, we Know that $U$ is an isometry.
    \begin{equation*}
        \begin{split}
            \|U\|_{op} = \mathop{max} \limits _ {\|v\| = 1, v \in \mathbb{C}^n} \|Uv\| = \|v\| = 1
        \end{split}
    \end{equation*}
\end{proof}

\subsubsection*{(b)}
By \textbf{Proposition 4.30}, we know that if $U$ is unitary, then all columns of $U$ is orthonomal. By \textbf{Theorem 4.3}, we know that all columns are independent. 
By \textbf{Theorem 3.28}, we know that the columns are orthonomal basis. By \textbf{Corrolary 4.30}, we Know that $U$ is an isometry.
\begin{equation*}
    \begin{split}
        \|U\|_F = \sqrt{trU^*U }= \sqrt{trI} = \sqrt{n}
    \end{split}
\end{equation*}



\end{document}