\documentclass{article}
\usepackage{indentfirst}
\usepackage{geometry}
\usepackage{ntheorem}
\usepackage{amsmath}
\usepackage{amssymb}
\newtheorem*{proposition}{Proposition}
\newtheorem*{definition}{Definition}
\newtheorem*{corrolary}{Corrolary}
\newtheorem*{consider}{Consider}
\newtheorem*{theorem}{Theorem}
\newtheorem*{suppose}{Suppose}
\newtheorem*{notice}{Notice}
\newtheorem*{define}{Define}
\newtheorem*{denote}{Denote}
\newtheorem*{lemma}{Lemma}
\newtheorem*{claim}{Claim}
\newtheorem*{proof}{Proof}
\newtheorem*{case}{Case}
%\begin{equation*} \end{equation*}%
%\begin{equation} \end{equation}%
%\begin{split} \end{split}%
%\begin{cases} \end{cases}%
%\subsubsection*{(a)}%
%\subsection*{(a)}%
%\textbf{}%
%\textit{}%
%\noindent%
%\paragraph{}%
%\page{}%
%\\%
%$\langle \rangle$%
%$\| \|$%
\geometry{a4paper}
\title{Homework10}
\author{Zhihao Wang} 
\date{04/08/2022}
\begin{document}
\maketitle 


\subsection*{4.1.4}
\begin{denote}
    The $jth$ col of $A, B$ is $A_j, B_j$. And $A_j, B_j \in \mathbb{C}^m$. So that, $\|A_j\|^2 = \langle A_j, A_j\rangle$, as well as for $B_j$.
\end{denote}
\begin{proof}
    By the definition of inner product $\sum_{i=1}^{i=m} [A]_{ij}[B]_{ij} =\langle A_j, B_j \rangle$.
    For the inner product of two matrices: $\langle A, B\rangle _{F} =\sum_{j=1}^{j=n} \sum_{i=1}^{i=m} [A]_{ij}[B]_{ij} =\sum_{j=1}^{j=n} \langle A_j, B_j \rangle$. 
    By the definition of the norm: $\|A\|^2 {_F}  = \langle A, A \rangle _F = \sum_{j=1}^{j=n} \langle A_j, A_j \rangle = \sum_{j=1}^{j=n}\|A_j\|^2$.
\end{proof}

\subsection*{4.1.8}
\noindent Suppose there exists $w \in V$. $\langle v - w, v - w \rangle = \langle w - v, w - v \rangle \Rightarrow \|v - w\| = \|w - v\|$
\begin{equation*}
    \begin{split}
          \|v\| = \|v - u + u\| \le \|v - u\| + \|u\| \Rightarrow &\|v - u\| \ge \|v\| - \|u\| = 11 - 2 = 9 \\
        & \|v - u\| \le \|v-w\| + \|w - u\| = \|v-w\| + \|u - w\| = 8
    \end{split}
\end{equation*}
Contradiction.

\subsection*{4.1.14}
\subsubsection*{(a)}
\begin{equation*}
    \begin{split}
        \frac 1 4 (\|v + w\|^2 - \|v - w\|^2) = \frac 1 4 (\langle v+w, v+w\rangle - \langle v-w, v-w \rangle) = \frac 1 4(2 \langle v, w \rangle - (-2) \langle v, w \rangle) = \langle v, w \rangle
    \end{split}
\end{equation*}

\subsubsection*{(b)}
\begin{equation*}
    \begin{split}
        &\frac 1 4 (\|v + w\|^2 - \|v - w\|^2 + i\|v + iw\|^2 - i\|v-iw\|^2) \\
        &= \frac 1 4 (\langle v+w, v+w\rangle - \langle v-w, v-w \rangle + i(\langle v+ iw, v+ iw\rangle - \langle v- iw, v- iw \rangle)) \\
        &= \frac 1 4(2 \langle w , v\rangle +2 \langle v, w \rangle + i(2i \langle w , v\rangle - 2i \langle v, w \rangle)) = \langle v, w \rangle
    \end{split}
\end{equation*}

\subsection*{4.1.14}
\begin{proof}
    $\forall v, w \in \mathbb{V}$. Use \textbf{Definition} and \textbf{Proposition 4.2.2}.
    \begin{equation*}
        \begin{split}
            & \langle 0, v \rangle = 0 * \langle w, v \rangle = 0 \\
            & \langle v, 0 \rangle = \overline 0 * \langle v, w \rangle = 0
        \end{split}
    \end{equation*}
\end{proof}

\subsection*{4.2.4}
\noindent This Exercise we need to ouse \textbf{Theorem 4.9} 

\subsubsection*{(a)}
\begin{equation*}
    \begin{split}
        & \langle \begin{bmatrix}
            1 \\ 2 \\ -3
        \end{bmatrix}, \frac{1}{ \sqrt{2}} \begin{bmatrix}
            1 \\ 0 \\ -1
        \end{bmatrix} \rangle = 2 \sqrt{2} \\
        & \langle \begin{bmatrix}
            1 \\ 2 \\ -3
        \end{bmatrix}, \frac{1}{ \sqrt{6}} \begin{bmatrix}
            1 \\ -2 \\ 1
        \end{bmatrix} \rangle = - \sqrt{6}
    \end{split}
    \Rightarrow \begin{bmatrix}
        2 \sqrt{2} \\ - \sqrt{6}
    \end{bmatrix}
\end{equation*}

\subsubsection*{(b)}
Refer to \textbf{Exercise 4.1.4}
\begin{equation*}
    \begin{split}
        & \langle \begin{bmatrix}
            5 & 7 \\7 & 2
        \end{bmatrix}, \frac{1}{2} \begin{bmatrix}
            1 & 1 \\ 1 & 1
        \end{bmatrix} \rangle = \frac{21}{2} \\
        & \langle \begin{bmatrix}
            5 & 7 \\7 & 2
        \end{bmatrix}, \frac{1}{2} \begin{bmatrix}
            1 & 1 \\ -1 & -1
        \end{bmatrix} \rangle = \frac{3}{2} \\
        & \langle \begin{bmatrix}
            5 & 7 \\7 & 2
        \end{bmatrix}, \frac{1}{2} \begin{bmatrix}
            1 & -1 \\ 1 & -1
        \end{bmatrix} \rangle = \frac{3}{2} \\
        & \langle \begin{bmatrix}
            5 & 7 \\7 & 2
        \end{bmatrix}, \frac{1}{2} \begin{bmatrix}
            1 & -1 \\ -1 & 1
        \end{bmatrix} \rangle = \frac{-7}{2} 
    \end{split}
    \Rightarrow \begin{bmatrix}
        \frac{21}{2} \\ \frac{3}{2} \\ \frac{3}{2} \\ \frac{-7}{2} 
    \end{bmatrix}
\end{equation*}

\subsubsection*{(c)}
\begin{equation*}
    \begin{split}
        & \langle
        \begin{bmatrix}
            -3 \\ 0 \\ 1 \\ 2
        \end{bmatrix}, \frac{1}{2} \begin{bmatrix}
            1 \\ 1 \\ 1 \\ 1
        \end{bmatrix}
        \rangle = 0 \\
        & \langle
        \begin{bmatrix}
            -3 \\ 0 \\ 1 \\ 2
        \end{bmatrix}, \frac{1}{2} \begin{bmatrix}
            1 \\ i \\ -1 \\ -i
        \end{bmatrix}
        \rangle = -2 + i \\
        & \langle
        \begin{bmatrix}
            -3 \\ 0 \\ 1 \\ 2
        \end{bmatrix}, \frac{1}{2} \begin{bmatrix}
            1 \\ -1 \\ 1 \\ -1
        \end{bmatrix}
        \rangle = -2 \\
        & \langle
        \begin{bmatrix}
            -3 \\ 0 \\ 1 \\ 2
        \end{bmatrix}, \frac{1}{2} \begin{bmatrix}
            1 \\ -i \\ -1 \\ i
        \end{bmatrix}
        \rangle = -2 - i
    \end{split}
    \Rightarrow \begin{bmatrix}
        0 \\ -2 + i \\ -2 \\ -2 - i
    \end{bmatrix}
\end{equation*}

\subsection*{4.2.6}
\noindent This question we will use \textbf{Theorem 4.9}. Assume the $ith$ row and $jth$ column of the matrix is the following equation: 
\begin{equation*}
    \begin{split}
        [[T]_{B_{\mathbb{V}}B_{\mathbb{V}}}]_{ij} = \langle T(e_j), e_i \rangle
    \end{split}
\end{equation*}

\subsubsection*{(a)}
\begin{equation*}
    \begin{split}
        &[[T]_{B_\mathbb{V}}]_{11} = \langle T(e_1), e_1 \rangle = \langle \frac{1}{\sqrt{2}} \begin{bmatrix}
            0 \\ -1 \\ 1
        \end{bmatrix}, \frac{1}{\sqrt{2}}\begin{bmatrix}
            1 \\ 0 \\ -1
        \end{bmatrix} \rangle = - \frac{1}{2}\\
        &[[T]_{B_\mathbb{V}}]_{21} = \langle T(e_1), e_2 \rangle =  \langle \frac{1}{\sqrt{2}} \begin{bmatrix}
            0 \\ -1 \\ 1
        \end{bmatrix}, \frac{1}{\sqrt{6}}\begin{bmatrix}
            1 \\ -2 \\ 1
        \end{bmatrix} \rangle = \frac{\sqrt{3}}{2}\\
        &[[T]_{B_\mathbb{V}}]_{12} = \langle T(e_2), e_1 \rangle =  \langle \frac{1}{\sqrt{6}} \begin{bmatrix}
            -2 \\ 1 \\ 1
        \end{bmatrix}, \frac{1}{\sqrt{2}}\begin{bmatrix}
            1 \\ 0 \\ -1
        \end{bmatrix} \rangle = - \frac{\sqrt{3}}{2}\\
        &[[T]_{B_\mathbb{V}}]_{22} = \langle T(e_2), e_2 \rangle =  \langle \frac{1}{\sqrt{6}} \begin{bmatrix}
            -2 \\ 1 \\ 1
        \end{bmatrix}, \frac{1}{\sqrt{6}}\begin{bmatrix}
            1 \\ -2 \\ 1
        \end{bmatrix} \rangle = - \frac{1}{2}
    \end{split}
    \Rightarrow \begin{bmatrix}
        - \frac{1}{2} & - \frac{\sqrt{3}}{2}\\ \frac{\sqrt{3}}{2} & - \frac{1}{2}
    \end{bmatrix}
\end{equation*}

\subsubsection*{(b)}
\begin{equation*}
    \begin{split}
        &
        \begin{bmatrix}
            \langle T(e_1), e_1\rangle & \langle T(e_2), e_1\rangle & \langle T(e_3), e_1\rangle & \langle T(e_4), e_1\rangle\\ 
            \langle T(e_1), e_2\rangle & \langle T(e_2), e_2\rangle & \langle T(e_3), e_2\rangle & \langle T(e_4), e_2\rangle\\ 
            \langle T(e_1), e_3\rangle & \langle T(e_2), e_3\rangle & \langle T(e_3), e_3\rangle & \langle T(e_4), e_3\rangle\\ 
            \langle T(e_1), e_4\rangle & \langle T(e_2), e_4\rangle & \langle T(e_3), e_4\rangle & \langle T(e_4), e_4\rangle\\ 
        \end{bmatrix} \\
        & = 
        \begin{bmatrix}
            1 & 0 & 0 & 0 \\
            0 & 0 & 1 & 0 \\
            0 & 1 & 0 & 0 \\
            0 & 0 & 0 & 1 \\
        \end{bmatrix}
    \end{split}
\end{equation*}

\subsubsection*{(c)}
\begin{equation*}
    \begin{split}
        &
        \begin{bmatrix}
            \langle T(e_1), e_1\rangle & \langle T(e_2), e_1\rangle & \langle T(e_3), e_1\rangle & \langle T(e_4), e_1\rangle\\ 
            \langle T(e_1), e_2\rangle & \langle T(e_2), e_2\rangle & \langle T(e_3), e_2\rangle & \langle T(e_4), e_2\rangle\\ 
            \langle T(e_1), e_3\rangle & \langle T(e_2), e_3\rangle & \langle T(e_3), e_3\rangle & \langle T(e_4), e_3\rangle\\ 
            \langle T(e_1), e_4\rangle & \langle T(e_2), e_4\rangle & \langle T(e_3), e_4\rangle & \langle T(e_4), e_4\rangle\\
        \end{bmatrix} \\
        & =
        \begin{bmatrix}
            1 & 0 & 0 & 0 \\
            0 & 0 & 0 & -i \\ 
            0 & 0 & -1 & 0 \\
            0 & i & 0 & 0 
        \end{bmatrix}
    \end{split}
\end{equation*}

\subsection*{4.2.8}







\end{document}
% \begin{equation*}
%     \begin{split}
%         & (\lfloor \frac{t}{d_3} \rfloor - 2p) + f((t \mod d_3) + 2 d_3 p, 2) \\
%         \rightarrow & (\lfloor \frac{t}{d_3} \rfloor - 2p) 
%         + \lfloor \frac{(t \mod d_3) + 2 d_3 p}{d_2} \rfloor 
%         + f(((t \mod d_3) + 2 d_3 p) \mod d_2, 1) \\
%         = & (\lfloor \frac{t}{25} \rfloor - 2p) 
%         + \lfloor \frac{(t \mod 25) + 2 \cdot 25 \cdot p}{10} \rfloor 
%         + f(((t \mod 25) + 2 \cdot 25 \cdot p) \mod 10, 1) \\
%         = & (\lfloor \frac{t}{25} \rfloor - 2p) 
%         + \lfloor \frac{t \mod 25}{10} \rfloor + 5 p 
%         + f((t \mod 25) \mod 10, 1) \\
%         = & \lfloor \frac{t}{25} \rfloor
%         + \lfloor \frac{t \mod 25}{10} \rfloor + 3 p 
%         + f((t \mod 25) \mod 10, 1) 
%     \end{split}
%     \end{equation*}

% \begin{equation}
%     \begin{split}
%     (a + b)^4
%       &= (a + b)^2 (a + b)^2     \\
%       &= (a^2 + 2ab + b^2)
%         (a^2 + 2ab + b^2)       \\
%       &= a^4 + 4a^3b + 6a^2b^2 + 4ab^3 + b^4
%     \end{split}
%    \end{equation}
