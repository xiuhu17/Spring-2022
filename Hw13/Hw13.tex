\documentclass{article}
\usepackage{indentfirst}
\usepackage{geometry}
\usepackage{ntheorem}
\usepackage{amsmath}
\usepackage{amssymb}
\newtheorem*{proposition}{Proposition}
\newtheorem*{definition}{Definition}
\newtheorem*{corrolary}{Corrolary}
\newtheorem*{consider}{Consider}
\newtheorem*{theorem}{Theorem}
\newtheorem*{suppose}{Suppose}
\newtheorem*{notice}{Notice}
\newtheorem*{define}{Define}
\newtheorem*{denote}{Denote}
\newtheorem*{lemma}{Lemma}
\newtheorem*{claim}{Claim}
\newtheorem*{proof}{Proof}
\newtheorem*{case}{Case}
%\begin{equation*} \end{equation*}%
%\begin{equation} \end{equation}%
%\begin{split} \end{split}%
%\begin{cases} \end{cases}%
%\subsubsection*{(a)}%
%\subsection*{(a)}%
%\textbf{}%
%\textit{}%
%\noindent%
%\paragraph{}%
%\page{}%
%\\%
%$\langle \rangle$%
%$\| \|$%
% \begin{equation*}
%     \begin{split}
%         & (\lfloor \frac{t}{d_3} \rfloor - 2p) + f((t \mod d_3) + 2 d_3 p, 2) \\
%         \rightarrow & (\lfloor \frac{t}{d_3} \rfloor - 2p) 
%         + \lfloor \frac{(t \mod d_3) + 2 d_3 p}{d_2} \rfloor 
%         + f(((t \mod d_3) + 2 d_3 p) \mod d_2, 1) \\
%         = & (\lfloor \frac{t}{25} \rfloor - 2p) 
%         + \lfloor \frac{(t \mod 25) + 2 \cdot 25 \cdot p}{10} \rfloor 
%         + f(((t \mod 25) + 2 \cdot 25 \cdot p) \mod 10, 1) \\
%         = & (\lfloor \frac{t}{25} \rfloor - 2p) 
%         + \lfloor \frac{t \mod 25}{10} \rfloor + 5 p 
%         + f((t \mod 25) \mod 10, 1) \\
%         = & \lfloor \frac{t}{25} \rfloor
%         + \lfloor \frac{t \mod 25}{10} \rfloor + 3 p 
%         + f((t \mod 25) \mod 10, 1) 
%     \end{split}
%     \end{equation*}

% \begin{equation}
%     \begin{split}
%     (a + b)^4
%       &= (a + b)^2 (a + b)^2     \\
%       &= (a^2 + 2ab + b^2)
%         (a^2 + 2ab + b^2)       \\
%       &= a^4 + 4a^3b + 6a^2b^2 + 4ab^3 + b^4
%     \end{split}
%    \end{equation}

\geometry{a4paper}
\title{Homework11}
\author{Zhihao Wang} 
\begin{document}
\maketitle 

\subsection*{5.3.6}
\begin{denote}
    The adjoint $B$ as matrix $K \in M_n\mathbb{(C)}$.
\end{denote}
$\forall x, y \in \mathbb{C}^n$
\begin{equation*}
    \begin{split}
         & \langle Bx, y \rangle_A = \langle ABx, Ay \rangle = \langle A^*ABx, y \rangle = \langle x, (A^*AB)^*y \rangle \\
         & \langle x, Ky \rangle_A = \langle Ax, AKy \rangle = \langle x, A^*AKy \rangle \\
         & \Rightarrow \langle x, ((A^*AB)^* - A^*AK)y \rangle = 0
    \end{split}
\end{equation*}
Since $A$ is invertible, to $A^*$ is invertible, so well as $A^*A$. By \textbf{Proposition 4.2.4}
\begin{equation*}
    \begin{split}
        ((A^*AB)^* - A^*AK)y = 0 \Rightarrow (A^*AB)^* - A^*AK = 0 \Rightarrow K = (A^*A)^{-1}B^*(A^*A)
    \end{split}
\end{equation*}

\subsection*{5.3.8}
\subsubsection*{(a)}
\begin{equation*}
    \begin{split}
        \langle D(p), q \rangle & = \int_{- \infty}^{\infty} p'(x)q(x)e^{- \frac{x^2}{2}} \,dx = \int_{- \infty}^{\infty} q(x)e^{- \frac{x^2}{2}} \,d p(x) \\
        & = p(x)q(x)e^{- \frac{x^2}{2}} |_{- \infty}^{\infty} - \int_{- \infty}^{\infty} p(x) \,d (q(x)e^{- \frac{x^2}{2}}) \\
        & = - \int_{- \infty}^{\infty} p(x) \,d (q(x)e^{- \frac{x^2}{2}})
        \\ & = - \int_{- \infty}^{\infty} p(x)(q'(x) - xq(x))e^{- \frac{x^2}{2}}dx
        \\ & = \langle p, D^*(q) \rangle
        \\ & \Rightarrow D^*(q(x)) =  -q'(x) + xq(x)
    \end{split}
\end{equation*}
\subsubsection*{(b)}
\begin{denote}
    Basis $(1, x, ..., x^{n-1})$ is $B_v$. And Basis $(1, x, ..., x^{n})$ is $B_w$
\end{denote}
The $ith$ col of matrix of $D$ with respect of $B_v, B_w$ is 
\begin{equation*}
    \begin{split}
        [D_i]_{B_w} = [D(x^{i-1})]_{B_w}
    \end{split}
\end{equation*}
Therefore, the matrix is $n*(n+1)$
\begin{equation*}
    \begin{split}
        D = \begin{bmatrix}
            0 & 1 & 0 & 0 & ... & 0 \\
            0 & 0 & 2 & 0 & ... & 0 \\
            0 & 0 & 0 & 3 & ... & 0 \\
            ... & ... & ... & ... & ... & ... \\
            0 & 0 & 0 & 0 & ... & n
        \end{bmatrix}
    \end{split}
\end{equation*}

\subsubsection*{(c)}
\begin{denote}
    Basis $(1, x, ..., x^{n})$ is $B_w$. And Basis $(1, x, ..., x^{n - 1})$ is $B_v$
\end{denote}
The $ith$ col of matrix of $D$ with respect of $B_w, B_v$ is 
\begin{equation*}
    \begin{split}
        [D^*_i]_{B_v} = [D^*(x^{i-1})]_{B_v} = [-(i-1)x^{i-2}+x^i]_{B_v}
    \end{split}
\end{equation*}
Therefore, the matrix is $(n+1)*n$
\begin{equation*}
    \begin{split}
        D = \begin{bmatrix}
            0 & -1 & 0 & 0 & ... & 0 \\
            1 & 0 & -2 & 0 & ... & 0 \\
            0 & 1 & 0 & -3 & ... & 0 \\
            0 & 0 & 1 & 0 & ... & 0 \\
            ... & ... & ... & ... & ... & ... \\
            0 & 0 & 0 & 0 & ... & 1
        \end{bmatrix}
    \end{split}
\end{equation*}

\subsubsection*{(d)}
Since the basis we use are not orthonomal basis. For example
\begin{equation*}
    \begin{split}
        \int_{- \infty}^{\infty} 1 * 1 * e^{- \frac{x^2}{2}}dx \neq 1 
    \end{split}
\end{equation*}

\subsection*{5.3.12}
Since it is a self adjoint Map, $T^* = T$.
We can rewrite v as following
\begin{equation*}
    \begin{split}
        v = T(v) + (v-T(v))
    \end{split}
\end{equation*}
Now, consider $U = rangeT$, and we need to prove that $T(v)$ is orthogonal to $v-T(v)$
\begin{equation*}
    \begin{split}
        \langle T(v), v-T(v) \rangle = \langle v, T^*(v-T(v)) \rangle = \langle v, T(v-T(v)) \rangle
        = \langle v, T(v)-T^2(v) \rangle = 0
    \end{split}
\end{equation*}
Therefore, $T(v)$ is orthogonal to $v-T(v)$, so we get $T$ is the orthogonal
projection onto $U = range T$.

\subsection*{5.3.18/19}
\begin{suppose}
    The random two distinct eignvalues are $\lambda_1, \lambda_2$, with two independent corrosponding eignvectors $v_1, v_2$.
\end{suppose}
\begin{equation*}
    \begin{split}
        \langle v_1 ,v_1 \rangle &= \frac{1}{\lambda_1} \langle T(v_1), v_1 \rangle = \frac{1}{\lambda_1} \langle v_1, T^*(v_1) \rangle\\
         &= -\frac{1}{\lambda_1} \langle v_1, T(v_1) \rangle = -\frac{\overline{\lambda_1}}{\lambda_1} \langle v_1, v_1 \rangle
         \\ 
    \end{split}
\end{equation*}
So we get
\begin{equation*}
    \begin{split}
        \overline{\lambda_1} = -\lambda_1 \Rightarrow  \overline{\lambda_1} + \lambda_1 = 0 = 2Re(\lambda_1)
    \end{split}
\end{equation*} 
Since the real part is zero, so it is purely imaginary. We can further prove that
\begin{equation*}
    \begin{split}
        \langle v_1 ,v_2 \rangle &= \frac{1}{\lambda_1} \langle T(v_1), v_2 \rangle = \frac{1}{\lambda_1} \langle v_1, T^*(v_2) \rangle\\
         &= -\frac{1}{\lambda_1} \langle v_1, T(v_2) \rangle = -\frac{\overline{\lambda_2}}{\lambda_1} \langle v_1, v_2 \rangle
    \end{split}
\end{equation*} 
\begin{denote}
    $\lambda_1 = b_1i, \lambda_2 = b_2 i$
\end{denote}
\begin{equation*}
    \begin{split}
        -\frac{\overline{\lambda_2}}{\lambda_1} = 1 \Rightarrow b_2i = b_1i
    \end{split}
\end{equation*}
\textbf{Contradiction}, so all of them are orthogonal.
\end{document}