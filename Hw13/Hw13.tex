\documentclass{article}
\usepackage{indentfirst}
\usepackage{geometry}
\usepackage{ntheorem}
\usepackage{amsmath}
\usepackage{amssymb}
\newtheorem*{proposition}{Proposition}
\newtheorem*{definition}{Definition}
\newtheorem*{corrolary}{Corrolary}
\newtheorem*{consider}{Consider}
\newtheorem*{theorem}{Theorem}
\newtheorem*{suppose}{Suppose}
\newtheorem*{notice}{Notice}
\newtheorem*{define}{Define}
\newtheorem*{denote}{Denote}
\newtheorem*{lemma}{Lemma}
\newtheorem*{claim}{Claim}
\newtheorem*{proof}{Proof}
\newtheorem*{case}{Case}
%\begin{equation*} \end{equation*}%
%\begin{equation} \end{equation}%
%\begin{split} \end{split}%
%\begin{cases} \end{cases}%
%\subsubsection*{(a)}%
%\subsection*{(a)}%
%\textbf{}%
%\textit{}%
%\noindent%
%\paragraph{}%
%\page{}%
%\\%
%$\langle \rangle$%
%$\| \|$%
% \begin{equation*}
%     \begin{split}
%         & (\lfloor \frac{t}{d_3} \rfloor - 2p) + f((t \mod d_3) + 2 d_3 p, 2) \\
%         \rightarrow & (\lfloor \frac{t}{d_3} \rfloor - 2p) 
%         + \lfloor \frac{(t \mod d_3) + 2 d_3 p}{d_2} \rfloor 
%         + f(((t \mod d_3) + 2 d_3 p) \mod d_2, 1) \\
%         = & (\lfloor \frac{t}{25} \rfloor - 2p) 
%         + \lfloor \frac{(t \mod 25) + 2 \cdot 25 \cdot p}{10} \rfloor 
%         + f(((t \mod 25) + 2 \cdot 25 \cdot p) \mod 10, 1) \\
%         = & (\lfloor \frac{t}{25} \rfloor - 2p) 
%         + \lfloor \frac{t \mod 25}{10} \rfloor + 5 p 
%         + f((t \mod 25) \mod 10, 1) \\
%         = & \lfloor \frac{t}{25} \rfloor
%         + \lfloor \frac{t \mod 25}{10} \rfloor + 3 p 
%         + f((t \mod 25) \mod 10, 1) 
%     \end{split}
%     \end{equation*}

% \begin{equation}
%     \begin{split}
%     (a + b)^4
%       &= (a + b)^2 (a + b)^2     \\
%       &= (a^2 + 2ab + b^2)
%         (a^2 + 2ab + b^2)       \\
%       &= a^4 + 4a^3b + 6a^2b^2 + 4ab^3 + b^4
%     \end{split}
%    \end{equation}

\geometry{a4paper}
\title{Homework11}
\author{Zhihao Wang} 
\begin{document}
\maketitle 

\subsection*{5.3.6}
\begin{denote}
    The adjoint $B$ as matrix $K \in M_n\mathbb{(C)}$.
\end{denote}
$\forall x, y \in \mathbb{C}^n$
\begin{equation*}
    \begin{split}
         & \langle Bx, y \rangle_A = \langle ABx, Ay \rangle = \langle A^*ABx, y \rangle = \langle x, (A^*AB)^*y \rangle \\
         & \langle x, Ky \rangle_A = \langle Ax, AKy \rangle = \langle x, A^*AKy \rangle \\
         & \Rightarrow \langle x, ((A^*AB)^* - A^*AK)y \rangle = 0
    \end{split}
\end{equation*}
Since $A$ is invertible, to $A^*$ is invertible, so well as $A^*A$. By \textbf{Proposition 4.2.4}
\begin{equation*}
    \begin{split}
        ((A^*AB)^* - A^*AK)y = 0 \Rightarrow (A^*AB)^* - A^*AK = 0 \Rightarrow K = (A^*A)^{-1}B^*(A^*A)
    \end{split}
\end{equation*}

\subsection*{5.3.8}
\subsubsection*{(a)}
\begin{equation*}
    \begin{split}
        \langle D(p), q \rangle & = \int_{- \infty}^{\infty} p'(x)q(x)e^{- \frac{x^2}{2}} \,dx = \int_{- \infty}^{\infty} q(x)e^{- \frac{x^2}{2}} \,d p(x) \\
        & = p(x)q(x)e^{- \frac{x^2}{2}} |_{- \infty}^{\infty} - \int_{- \infty}^{\infty} p(x) \,d (q(x)e^{- \frac{x^2}{2}}) \\
        & = - \int_{- \infty}^{\infty} p(x) \,d (q(x)e^{- \frac{x^2}{2}})
        \\ & = - \int_{- \infty}^{\infty} p(x)(q'(x) - xq(x))e^{- \frac{x^2}{2}}dx
        \\ & = \langle p, D^*(q) \rangle
        \\ & \Rightarrow D^*(q(x)) =  -q'(x) + xq(x)
    \end{split}
\end{equation*}
\subsubsection*{(b)}
\begin{denote}
    Basis $(1, x, ..., x^{n-1})$ is $B_v$. And Basis $(1, x, ..., x^{n})$ is $B_w$
\end{denote}
The $ith$ col of matrix of $D$ with respect of $B_v, B_w$ is 
\begin{equation*}
    \begin{split}
        [D_i]_{B_w} = [D(x^{i-1})]_{B_w}
    \end{split}
\end{equation*}
Therefore, the matrix is $n*(n+1)$
\begin{equation*}
    \begin{split}
        D = \begin{bmatrix}
            0 & 1 & 0 & 0 & ... & 0 \\
            0 & 0 & 2 & 0 & ... & 0 \\
            0 & 0 & 0 & 3 & ... & 0 \\
            ... & ... & ... & ... & ... & ... \\
            0 & 0 & 0 & 0 & ... & n
        \end{bmatrix}
    \end{split}
\end{equation*}

\subsubsection*{(c)}
\begin{denote}
    Basis $(1, x, ..., x^{n})$ is $B_w$. And Basis $(1, x, ..., x^{n - 1})$ is $B_v$
\end{denote}
The $ith$ col of matrix of $D$ with respect of $B_w, B_v$ is 
\begin{equation*}
    \begin{split}
        [D^*_i]_{B_v} = [D^*(x^{i-1})]_{B_v} = [-(i-1)x^{i-2}+x^i]_{B_v}
    \end{split}
\end{equation*}
Therefore, the matrix is $(n+1)*n$
\begin{equation*}
    \begin{split}
        D = \begin{bmatrix}
            0 & -1 & 0 & 0 & ... & 0 \\
            1 & 0 & -2 & 0 & ... & 0 \\
            0 & 1 & 0 & -3 & ... & 0 \\
            0 & 0 & 1 & 0 & ... & 0 \\
            ... & ... & ... & ... & ... & ... \\
            0 & 0 & 0 & 0 & ... & 1
        \end{bmatrix}
    \end{split}
\end{equation*}

\subsubsection*{(d)}
Since the basis we use are not orthonomal basis. For example
\begin{equation*}
    \begin{split}
        \int_{- \infty}^{\infty} 1 * 1 * e^{- \frac{x^2}{2}}dx \neq 1 
    \end{split}
\end{equation*}

\subsection*{5.3.12}
Since it is a self adjoint Map, $T^* = T$.
We can rewrite v as following
\begin{equation*}
    \begin{split}
        v = T(v) + (v-T(v))
    \end{split}
\end{equation*}
Now, consider $U = rangeT$, and we need to prove that $T(v)$ is orthogonal to $v-T(v)$
\begin{equation*}
    \begin{split}
        \langle T(v), v-T(v) \rangle = \langle v, T^*(v-T(v)) \rangle = \langle v, T(v-T(v)) \rangle
        = \langle v, T(v)-T^2(v) \rangle = 0
    \end{split}
\end{equation*}
Therefore, $T(v)$ is orthogonal to $v-T(v)$, so we get $T$ is the orthogonal
projection onto $U = range T$.

\subsection*{5.3.18/19}
\begin{suppose}
    The random two distinct eignvalues are $\lambda_1, \lambda_2$, with two independent corrosponding eignvectors $v_1, v_2$.
\end{suppose}
\begin{equation*}
    \begin{split}
        \langle v_1 ,v_1 \rangle &= \frac{1}{\lambda_1} \langle T(v_1), v_1 \rangle = \frac{1}{\lambda_1} \langle v_1, T^*(v_1) \rangle\\
         &= -\frac{1}{\lambda_1} \langle v_1, T(v_1) \rangle = -\frac{\overline{\lambda_1}}{\lambda_1} \langle v_1, v_1 \rangle
         \\ 
    \end{split}
\end{equation*}
So we get
\begin{equation*}
    \begin{split}
        \overline{\lambda_1} = -\lambda_1 \Rightarrow  \overline{\lambda_1} + \lambda_1 = 0 = 2Re(\lambda_1)
    \end{split}
\end{equation*} 
Since the real part is zero, so it is purely imaginary. We can further prove that
\begin{equation*}
    \begin{split}
        \langle v_1 ,v_2 \rangle &= \frac{1}{\lambda_1} \langle T(v_1), v_2 \rangle = \frac{1}{\lambda_1} \langle v_1, T^*(v_2) \rangle\\
         &= -\frac{1}{\lambda_1} \langle v_1, T(v_2) \rangle = -\frac{\overline{\lambda_2}}{\lambda_1} \langle v_1, v_2 \rangle
    \end{split}
\end{equation*} 
\begin{denote}
    $\lambda_1 = b_1i, \lambda_2 = b_2 i$
\end{denote}
\begin{equation*}
    \begin{split}
        -\frac{\overline{\lambda_2}}{\lambda_1} = 1 \Rightarrow b_2i = b_1i
    \end{split}
\end{equation*}
\textbf{Contradiction}, so all of them are orthogonal.

\subsection*{5.4.2}
\subsubsection*{(a)}
First check whether it is hermitian matrix.
\begin{equation*}
\begin{split}
    \begin{bmatrix}
        7 & 6 \\ 6 & -2
    \end{bmatrix}^* = \begin{bmatrix}
        7 & 6 \\ 6 & -2
    \end{bmatrix}
\end{split}
\end{equation*}
So it is a hermitian. So all the eignvalues are real.
\begin{equation*}
    \begin{split}
        &\begin{bmatrix}
            7 - \lambda& 6  \\ 6  & -2 - \lambda
        \end{bmatrix} \stackrel{RREF}{\longrightarrow} \begin{bmatrix}
            7 - \lambda& 6  \\ 0  &  \lambda ^ 2 - 5 \lambda - 50
        \end{bmatrix} 
        \Rightarrow \lambda_1 = 10, \lambda_2 = -5
    \end{split}
\end{equation*}
So we get two eignvalues, with corrosponding eignvectors\begin{equation*}
    \begin{split}
        \begin{bmatrix}
            2 \\ 1
        \end{bmatrix}, \begin{bmatrix}
            1 \\ -2
        \end{bmatrix}
    \end{split}
\end{equation*}
We can get the corrosponding othonormal basis 
\begin{equation*}
    \begin{split}
        \frac{1}{\sqrt{5}}
        \begin{bmatrix}
            2 \\ 1
        \end{bmatrix},
        \frac{1}{\sqrt{5}}
        \begin{bmatrix}
            1 \\ -2
        \end{bmatrix}
    \end{split}
\end{equation*}
Finally,
\begin{equation*}
    \begin{split}
        \begin{bmatrix}
            7 & 6 \\ 6 & -2
        \end{bmatrix} =  \begin{bmatrix}
            \frac{2}{\sqrt{5}} & \frac{1}{\sqrt{5}} \\ \frac{1}{\sqrt{5}} & \frac{-2}{\sqrt{5}}
        \end{bmatrix}\begin{bmatrix}
            10 & 0 \\ 0 & -5 
        \end{bmatrix} \begin{bmatrix}
            \frac{2}{\sqrt{5}} & \frac{1}{\sqrt{5}} \\ \frac{1}{\sqrt{5}} & \frac{-2}{\sqrt{5}}
        \end{bmatrix}
    \end{split}
\end{equation*}

\subsubsection*{(b)}
We find it is a normal matrix. Now we need to find 
\begin{equation*}
    \begin{split}
        \begin{bmatrix}
            1 - \lambda & i \\ i & 1 - \lambda
        \end{bmatrix} \stackrel{RREF}{\longrightarrow} \begin{bmatrix}
            1 - \lambda & i \\ 0 & \lambda^2 - 2\lambda + 2
        \end{bmatrix} \Rightarrow \lambda_1 = 1 + i, \lambda_2 = 1 - i 
    \end{split}
\end{equation*}
So the corrosponding two eignvectors are
\begin{equation*}
    \begin{split}
        \frac{1}{\sqrt{2}}\begin{bmatrix}
            1 \\ 1
        \end{bmatrix},
        \frac{1}{\sqrt{2}}\begin{bmatrix}
            1 \\ -1
        \end{bmatrix}
    \end{split}
\end{equation*}
Finally, we get 
\begin{equation*}
    \begin{split}
        \begin{bmatrix}
            1  & i \\ i & 1 
        \end{bmatrix} = \begin{bmatrix}
            \frac{1}{\sqrt{2}} & \frac{1}{\sqrt{2}} \\ \frac{1}{\sqrt{2}} & \frac{-1}{\sqrt{2}}
        \end{bmatrix}\begin{bmatrix}
            1 + i & 0 \\ 0 & 1 - i
        \end{bmatrix}\begin{bmatrix}
            \frac{1}{\sqrt{2}} & \frac{1}{\sqrt{2}} \\ \frac{1}{\sqrt{2}} & \frac{-1}{\sqrt{2}}
        \end{bmatrix}
    \end{split}
\end{equation*}

\subsubsection*{(c)}
We find it is a normal matrix. Now we need to find 
\begin{equation*}
    \begin{split}
        &\begin{bmatrix}
            1 - \lambda & 0 & 0 \\ 0 & -\lambda & 2 \\ 0 & 2 & -\lambda
        \end{bmatrix} \stackrel{RREF}{\longrightarrow} \begin{bmatrix}
            1 - \lambda & 0 & 0 \\ 0 & -\lambda & 2 \\ 0 & 0 & \lambda^2 - 4
        \end{bmatrix}\\ & \Rightarrow \lambda_1 = 1, \lambda_1 = 2, \lambda_1 = -2
    \end{split} 
\end{equation*}
So the corrosponding eignvectors are\begin{equation*}
    \begin{split}
        \begin{bmatrix}
            1 \\ 0 \\ 0
        \end{bmatrix}, \frac{1}{\sqrt{2}} \begin{bmatrix}
            0 \\ 1 \\ 1
        \end{bmatrix},\frac{1}{\sqrt{2}} \begin{bmatrix}
            0 \\ -1 \\ 1
        \end{bmatrix}
    \end{split}
\end{equation*}
Finally we get 
\begin{equation*}
    \begin{split}
        \begin{bmatrix}
            1 & 0 & 0 \\ 0 & 0 & 2 \\ 0 & 2 & 0
        \end{bmatrix} = \begin{bmatrix}
            1 & 0 &0  \\ 0 & \frac{1}{\sqrt{2}} & \frac{-1}{\sqrt{2}} \\ 0 & \frac{1}{\sqrt{2}} & \frac{1}{\sqrt{2}}
        \end{bmatrix} \begin{bmatrix}
            1 & 0 & 0 \\ 0 & 2 & 0 \\0 & 0 & -2 
        \end{bmatrix}\begin{bmatrix}
            1 & 0 &0  \\ 0 & \frac{1}{\sqrt{2}} & \frac{1}{\sqrt{2}} \\ 0 & \frac{-1}{\sqrt{2}} & \frac{1}{\sqrt{2}}
        \end{bmatrix}
    \end{split}
\end{equation*}
\subsection*{5.4.6}

\noindent Since A is hermitian matrix, and is invertible, so all eignvalue of A are nonzero.
\begin{suppose}
    $A = Udiag(\lambda_1, ..., \lambda_n)U^*, \forall i \in {1, ..., n}, \lambda_i \ne 0$
\end{suppose}
Refer to the equation on \textbf{Page 324}
\begin{equation*}
    \begin{split}
        f(A) = Udiag(f(\lambda_1), ... , f(\lambda_n))U^* \Rightarrow \frac{1}{A} = Udiag(\frac{1}{\lambda _1}, ... , \frac{1}{\lambda _n})U^*
    \end{split} 
\end{equation*}
For $A^{-1}$, we get\begin{equation*}
    \begin{split}
        A^{-1} =  (Udiag(\lambda_1, ..., \lambda_n)U^*)^{-1} = Udiag(\frac{1}{\lambda _1}, ... , \frac{1}{\lambda _n})U^*
    \end{split}
\end{equation*}
Therefore, we get\begin{equation*}
    \begin{split}
        \frac{1}{A} = A^{-1}
    \end{split}
\end{equation*}

\subsection*{5.4.10}
\subsubsection*{(a)}
We need to prove \textbf{5.4.9} first. 
\begin{suppose}
    $B = Udiag(\lambda_1, ..., \lambda_n)U^*, \forall i \in {1, ..., n}, \lambda_i \ge 0$
\end{suppose}
If every eignvalue of Hermitian matrix is nonnegative.
\begin{equation*}
    \begin{split}
        \langle Bx, x \rangle = \langle Udiag(\lambda_1, ..., \lambda_n)U^* x,x\rangle = \sum_{i = 1}^{i = n} \lambda_i | U^*x|^2 \ge 0
    \end{split}
\end{equation*}
if $\langle Bx, x \rangle \ge 0, \forall x \in V$, when we set x as its eignvectors, 
\begin{equation*}
    \begin{split}
        \langle Bx, x \rangle \ge 0 \Rightarrow \lambda_i\|x_i\|^2 \ge 0 \Rightarrow \lambda_i \ge 0
    \end{split}
\end{equation*} 
Now for \textbf{5.4.10}, we can prove that $A^*A$ is hermitian.
\begin{equation*}
    \begin{split}
        (A^*A)^* = A^*A
    \end{split}
\end{equation*}
So we can get \begin{equation*}
    \begin{split}
        \langle A^*Ax, x \rangle = \langle Ax, Ax \rangle = \|Ax\|^2 \ge 0
    \end{split}
\end{equation*}
So $A^*A$ is positive semidefinite.
\subsubsection*{(b)}
\textbf{FIRST METHOD:}
Since $rankA = n$
\begin{equation*}
    \begin{split}
        nullA = n - rankA = 0
    \end{split}
\end{equation*}
Therefore, A is injective. 
\begin{equation*}
    \begin{split}
        Ax = 0 \iff x = 0
    \end{split}
\end{equation*}
If, $y \in kerA^*A$, by \textbf{Proposition 5.16}
\begin{equation*}
    \begin{split}
        A^*Ay = 0 \iff Ay = 0, or \  Ay \in kerA^* \iff Ay = 0, or \  Ay \in (rangeA)^{\bot}
    \end{split}
\end{equation*}
Because , $A \ne 0$ So we get\begin{equation*}
    \begin{split}
        \langle Ay, Ay \rangle = 0 \iff \|Ay\|^2 = 0 \iff y = 0
    \end{split}
\end{equation*}
So $0$ can not be eignvalue of A,  all the eignvalue is positive. By \textbf{Conclusion from (a)} it is positive definite. \\ \\ 

\textbf{SECOND METHOD:}
Since $rankA = n$
\begin{equation*}
    \begin{split}
        nullA = n - rankA = 0
    \end{split}
\end{equation*}
Therefore, A is injective. 
\begin{equation*}
    \begin{split}
        Ax = 0 \iff x = 0
    \end{split}
\end{equation*}
So we can conclude that \begin{equation*}
    \begin{split}
        A^*Ay = 0 \Rightarrow \langle A^*Ay, y \rangle = 0 \iff \|Ay\|^2 = 0 \iff y = 0
    \end{split}
\end{equation*}
So $0$ can not be eignvalue of A,  all the eignvalue is positive. By \textbf{Conclusion from (a)} it is positive definite. 

\subsection*{5.4.22}
\begin{suppose}
    The random two distinct eignvaluesof $A$ are $\lambda_1, \lambda_2$, with two independent corrosponding eignvectors $x_1, x_2$, with 
    coordinate $v_1, ..., v_n$ and $w_1, ..., w_n$. Write $e_1, ..., e_n$ as standard basis, which is also an orthonomal basis.
\end{suppose}
\begin{lemma}
    Suppose
    \begin{equation*}
        \begin{split}
            & Dx_1 = \sum_{i = 1}^{i = n} v_i \lambda_i e_i = \lambda_1x_1 = \sum_{i = 1}^{i = n} v_i \lambda_1 e_i \\
            & \Rightarrow \sum_{i = 1}^{i = n} v_i (\lambda_i - \lambda_1) e_i = 0
        \end{split}
    \end{equation*}
    Since $e_1, ..., e_n$ are independent, so that $v_i = 0$ when $\lambda_i \ne \lambda_1$. These statement also holds true for $x_2$. So we get 
    \begin{equation*}
        \begin{split}
            x_1 = \stackrel{\lambda_i = \lambda_1}{\sum} v_ie_i, x_2 =  \stackrel{\lambda_i = \lambda_2}{\sum} 2_ie_i
        \end{split}
    \end{equation*}
    Since $\{e_i | \lambda_i = \lambda_1\} \cap \{e_i | \lambda_i = \lambda_2\} = \emptyset$.
    We get \begin{equation*}
        \begin{split}
            \langle x_1, x_2 \rangle = \langle \stackrel{\lambda_i = \lambda_1}{\sum} v_ie_i, \stackrel{\lambda_i = \lambda_2}{\sum} 2_ie_i\rangle = 0
        \end{split}
    \end{equation*}
    Therefore, eignvectors with different eignvalues of diagonal matrix is orthogonal.
\end{lemma}

Since $A$ is a noormal matrix, by \textbf{Theorem 5.23}we can rewrite it as $A = UDU^*$, with $U, U^*$ is an unitary matrix.
\begin{equation*}
    \begin{split}
       & Ax_1 = UDU^*x_1 = \lambda_1 x_1 \Rightarrow D(U^*x_1) = \lambda_1(U^*x_1) \\
        & Ax_2 = UDU^*x_2 = \lambda_2 x_2 \Rightarrow D(U^*x_2) = \lambda_2(U^*x_2)
    \end{split}
\end{equation*}
By \textbf{Theorem 4.25}
\begin{equation*}
    \begin{split}
        \langle x_1, x_2 \rangle = \langle U^*(x_1), U^*(x_2) \rangle
    \end{split}
\end{equation*}
Consider $U^*(x_1), U^*(x_2)$ as eignvectors of $D$ with different eignvalues $\lambda_1, \lambda_2$. 
By the \textbf{Lemma}, we know that eignvec with diff eignvalues are orthogonal.
\begin{equation*}
    \begin{split}
        \langle x_1, x_2 \rangle = \langle U^*(x_1), U^*(x_2) \rangle = 0
    \end{split}
\end{equation*} 
So, eignvectors with different eignvalues of normal matrix is orthogonal

\end{document}